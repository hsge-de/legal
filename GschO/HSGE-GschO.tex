\documentclass[10pt, twocolumn, parskip=half]{scrartcl}
\usepackage{fontspec}
\usepackage{polyglossia}
\usepackage{soul}

\usepackage{lastpage}
\usepackage{scrpage2}
\usepackage{geometry}
\usepackage{setspace}

\usepackage{tabu}
\usepackage{booktabs}

\usepackage{enumitem}

\usepackage{microtype}

\usepackage{titlesec}

\geometry{
	a4paper,
	left=1.5cm,
	right=1.5cm,
	top=3cm,
	bottom=3cm
}

\usepackage[defaultlines=2,all]{nowidow}
\usepackage{xifthen}

%\usepackage[switch,columnwise]{lineno}
%\linenumbers

\setmainlanguage{german}
\setdefaultlanguage[spelling=new, babelshorthands=true]{german}
\usepackage{csquotes}

\usepackage{graphicx}

\setmainfont[Numbers={Proportional,Lining}]{EB Garamond}
\setmainfont{Source Serif Pro}
\setsansfont{Source Sans Pro}
\setmonofont{Source Code Pro}

\onehalfspacing

\usepackage{titlesec}

\titleformat{\section}{\normalfont\large\sffamily\bfseries}{\space\thesection}{1em}{}

\titleformat{\subsection}{\normalfont\sffamily\bfseries}{\space\thesubsection}{1em}{}
\renewcommand{\thesubsection}{\Roman{subsection}} 

\setlength{\parindent}{0pt}

\renewcommand{\labelenumi}{(\theenumi)}
\renewcommand{\labelenumii}{(\theenumii)}

\begin{document}
	
\twocolumn[
{\Large\bfseries\sffamily Geschäftsordnung der Hochschulsegelgruppe Erlangen e. V.}\\
{\large\sffamily\bfseries Ausgabe 2019-1}
\vspace{0.5cm}\\]



\section*{Teil 1 – Angelegenheiten der Mitgliederversammlung}
\subsection{Vorstand und Sitz des Vereins}
\begin{enumerate}[noitemsep]
	\item Der Sitz des Vereins ist Erlangen, die Postanschrift des Vereins ist die Postanschrift des 1. Vorsitzenden.
	\item Schriftliche Nachrichten an den 1. Vorsitzenden und/oder den 2. Vorsitzenden gelten als dem Verein zugestellt.
	\item Vorsitzender und die 2. Vorsitzenden können jeweils einzeln den Verein nach außen und innen vertreten, der Kassierer im Bereich der Beitrags- und Versicherungsabrechnungen, der Schriftführer ist zuständig für die Protokolle.
\end{enumerate}

\subsection{Mitglieder}
\begin{enumerate}[noitemsep]
	\item Laut Satzung kann jede Person Mitglied werden. Die Mitgliedschaft bedeutet nicht automatisch eine Anerkennung als Skipper oder Ausbildungsskipper für die HSGE. Regelungen dazu finden sich in Teil 2.
	\item Ruht eine Mitgliedschaft, so ist das Mitglied nicht stimmberechtigt.
	\item Spätestens zum Beginn eines jeden neuen Geschäftsjahres muss dieser Beschluss überprüft werden.
	\item Die jeweils aktuelle Liste der Vereinsmitglieder wird den Ausschüssen und den Kursleitern unter der Maßgabe zur Verfügung gestellt, dass sie diese ausschließlich für Vereinszwecke
	verwenden.
\end{enumerate}

\subsection{Jahresbeitrag}
\begin{enumerate}[noitemsep]
	\item Die Jahresbeiträge sind abhängig von den Versicherungsbeträgen, den Aufgaben des Vereins und den Mitgliedsbeiträgen der Verbände. Sie sind zuletzt auf der Mitgliederversammlung vom 21.02.2008 beraten und beschlossen worden.
	\item Der Jahresbeitrag beträgt für:
		\begin{enumerate}[noitemsep]
			\item Einzelmitgliedschaft\dotfill \textbf{EURO 60,00}
			\item Familienmitgliedschaft\dotfill \textbf{EURO 80,00}
			\item Ehrenmitgliedschaft\dotfill \textbf{beitragsfrei}\\
			Ehrenmitglieder können gleichzeitig Mitglieder sein.
			\item Segler in Ausbildung\dotfill \textbf{mit Kursgebühr abgegolten}
			\item Segler mit ruhender Mitgliedschaft\dotfill \textbf{kein\\Beitrag}
			\item Befristet Mitgliedschaft für 4 Monate \\ohne Rundschreiben\dotfill \textbf{EURO 30,00}
		\end{enumerate}
\end{enumerate}

\subsection{Kasse}
\begin{enumerate}[noitemsep]
	\item Die Kassenprüfung findet einmal im Jahr statt. Der Termin dafür soll Ende Januar sein.
	
	Inhalt der Prüfung sind:
		\begin{itemize}[noitemsep]
			\item fand eine satzungsgemäße Verwendung der Mittel statt
			\item wurde wirtschaftlich gearbeitet
			\item wurden geeignete Aufzeichnungen geführt (s.u.).
		\end{itemize}
	\item Gegenstand der Prüfung sind geeignete Aufzeichnungen, in denen Einnahmen und Ausgaben, so wie deren jeweiliger Grund nachvollziehbar dargestellt sind.
	\item Eingehendere Prüfungen können durch übereinstimmenden Wunsch beider Kassenprüfer veranlasst werden.
	\item Das Ergebnis der Prüfung wird durch beide Kassenprüfer gegenüber der Mitgliederversammlung vertreten.	
\end{enumerate}

\subsection{Verträge}
\begin{enumerate}[noitemsep]
	\item  Verträge für den Verein (auch Charterverträge) dürfen nur Mitglieder des Vorstands oder von ihm Beauftragte nach den Vorgaben von Satzung und Geschäftsordnung abschließen.
	\item Der Vorstand gibt sich einen Geschäftsverteilungsplan, der veröffentlicht wird.
	\item Der Vorstand kann zur Erfüllung seiner Aufgaben Anordnungen treffen, die den Mitgliedern
	zur Kenntnis gegeben werden müssen.
\end{enumerate}

\subsection{Vereinseigene Schiffe}
\begin{enumerate}[noitemsep]
	\item Für die Nutzung von Vereinsschiffen wird eine Nutzergemeinschaft gegründet. Vereinsmitglieder können die Schiffe nutzen nach Anmeldung und einer Einweisung.
	\item Die HSGE stellt ihren Mitgliedern ein Schiff gegen Gebühr zur Verfügung. Arbeiten am Schiff können verrechnet werden.
	\item Die Nutzung wird vom Vorstand geregelt.
\end{enumerate}

\section*{Teil 2 - Führung der laufenden Geschäfte des Vereins durch den Vorstand}
\setcounter{subsection}{0}
\subsection{Geschäftsverteilung Vorstand}
Zu Beginn jeder Wahlperiode legt der Vorstand seine Geschäftsverteilung fest, die er den Mitgliedern zur Kenntnis bringt.


\subsection{Anweisungen des Vorstands}
\begin{enumerate}[noitemsep]
	\item Er kann in Erfüllung seiner Aufgaben nach § 10 Satz 1 a der Satzung auch kurzfristig Anordnungen erlassen, die als sogenannte „Segelanweisung Nr. ..“ verbindlich zu befolgende Regelungen darstellen.
	\item Die Segelanweisungen werden als Sammlung zusammengefasst.
	\item Widerspruch dagegen kann nach den Vorgaben der Satzung § 7 Satz 9 und analog § 8 Satz 1 f eingelegt werden.
	\item Sie gelten in jeden Fall ohne Einschränkung bis zur Aufhebung oder Änderung durch die Mitgliederversammlung oder den Vorstand.
	\item Um den Ausbildungs- und Leistungsstand der Vereinsmitglieder zu verbessern und hier insbesondere die, die in der Ausbildung von Vereinsmitgliedern aktiv sind oder aktiv werden, können diese kostenfrei, bzw. gegen eine ermäßigte Kursgebühr an geplanten und durchgeführten Kursen des Vereins teilnehmen. Die Kosten für extern bezogenes Lehrmaterial trägt der Kursteilnehmer.
	\item Mitglieder können die Teilnahme an den Kursen beim Vorstand beantragen. Über den Umfang dieses Kursangebots, als auch die Höhe der Ermäßigung, entscheidet der Vorstand nach eigenem Ermessen.
	\item Er informiert über diese Möglichkeit die Mitglieder.
\end{enumerate}


\section*{Regelungen zur Durchführung von Törns}
\subsection{Durchführung von Mitglieder-, Kojen-, Vereins- und Ausbildungstörns}
\begin{enumerate}[noitemsep]
	\item Die Durchführung von Vereins- und Ausbildungstörns, sowie von Theoriekursen gehört zu den laufenden Aufgaben des Vorstands. Diese Aufgaben kann er an geeignete Mitglieder delegieren.
	\item Für jedes Ausbildungsschiff wird der verantwortliche Ausbildungsskipper durch den Vorstand rechtzeitig bestimmt.
	\item Ein Wechsel des Ausbildungsskippers kann nur mit Zustimmung des Vorstands erfolgen.
	\item Er leistet dabei die notwendige Unterstützung.
\end{enumerate}

\subsection{Berufung zum Skipper und Ausbildungsskipper, Kursleiter}
\begin{enumerate}[noitemsep]
	\item Personaldiskussionen werden nicht öffentlich geführt.
	\item Der Vorstand stimmt mit den Ausbildungsskippern die Besetzung der Schiffe ab.
	\item Die letzte Entscheidung hat der gewählte Vorstand.
\end{enumerate}

\subsection{Skipper und Ausbildungsskipper}
\begin{enumerate}[noitemsep]
	\item Skipper und Ausbildungsskipper benötigen als Mindestanforderung die Fahrtberechtigungen der Bundesrepublik Deutschland (Binnen: SBF-Binnen, Meer: SBF-See), den BR- /SKS- oder höheren Schein oder vergleichbare Zertifikate anderer Staaten im Bereich See und ausreichend praktische Erfahrung, dazu ein gültiges Funkzeugnis.
	\item Der Vorstand kann ein Mitglied als Skipper oder Ausbildungsskipper anerkennen, wenn 2 aktive Mitglieder dem Vorstand gegenüber eine Empfehlung aussprechen.
	\item Alle Skipper und Ausbildungsskipper verpflichten sich, sich über die seemännischen, technischen, juristischen und ausbildungsspezifischen Entwicklungen im Bereich des Segelns fortlaufend zu informieren und weiter zu bilden.
	\item (a) Wer dieser Verpflichtung nicht nachkommt, kann auf Antrag vom Vorstand aus dem Kreis der Skipper ausgeschlossen werden. (b) Dies ist insbesondere dann gegeben, wenn ein Skipper oder Ausbildungsskipper
	\begin{enumerate}[noitemsep]
		\item  durch sein Verhalten auf Törns Anlass zur Klage über die Schiffsführung gibt.
		\item seinen Beitragsleistungen für die gemeinsamen Risikoversicherungen im laufenden Geschäftsjahr nicht nachkommt.
		
		Weiterhin kann ein Ausbildungskipper zum Skipper heruntergestuft werden, wenn er
		\item 3 mal einer Mitgliederversammlung ohne ausreichende Entschuldigung fern bleibt.
		\item nicht an Weiterbildungsmaßnahmen oder Ausbildungstörns im Verein oder außerhalb davon in regelmäßigen Abständen teilnimmt oder
		\item sich nicht regelmäßig an der Organisation und Durchführung von Ausbildungstörns beteiligt.
	\end{enumerate}
	\item Gegen die Aberkennung des Status als Skipper oder Ausbildungsskipper kann der oder die Betroffene in Übereinstimmung mit der Satzung § 12, Abschnitt 3, Satz 5 und 6 Einspruch beim Vorstand einlegen. Auch das weitere Verfahren entspricht diesem § 12, Satzung.
	\item Zusätzliche Aufgaben von Ausbildungsskippern:
	\begin{enumerate}[noitemsep]
		\item Sie bereiten gemeinsam mit dem Vorstand Ausbildungstörns vor, wählen die Schiffe, das Revier, und die Fahrtroute aus.
		\item Sie sorgen für eine fundierte Ausbildung der Teilnehmer nach den Vorgaben des DSV und seiner Prüfungsausschüsse. Dabei wissen sie, dass die verantwortungsbewusste Führung eines Schiffes und seiner Crew ein wesentlicher Bestandteil der Ausbildung ist.
		\item Sie legen die Zusammensetzung der Crews fest.
		\item Sie melden die Prüfungsteilnehmer beim zuständigen Prüfungsausschuss unter Vorlage der nötigen Unterlagen termingerecht an.
		\item Sie halten während des Törns untereinander und mit dem Prüfungsausschuss Kontakt.
		\item Ausbildungsskipper können ermächtigt oder verpflichtet werden, Törns zur Reviererkundung durchzuführen und den Skippern und Mitgliedern in geeigneter Form ihre Erkenntnisse und Erfahrungen mitzuteilen.
	\end{enumerate}
	\item Versicherungsbeiträge\\
	Die Charterkosten erhöhen sich für jeden Segelschüler um den Jahresbeitrag, der an die HSGE e.V. abgeführt wird, sofern sie nicht HSGE-Mitglieder sind. Jeder Teilnehmer einer Schulungsmaßnahme der HSGE ist anschließend für mindestens 1 Jahr Mitglied.
\end{enumerate}

\subsection{Sonstiges zur Vorbereitung von Törns}
\begin{enumerate}[noitemsep]
	\item Nach den Angaben zur Törnplanung im Frühjahr werden die einzelnen Ausbildungstörns durch den Verein versichert.
	\item Jeder versicherte Skipper erhält eine Dokumentenmappe, die wesentliche Daten und Informationen enthält. Die Mappe bleibt Eigentum des Vereins und ist bei Beendigung der Mitgliedschaft an den Verein zurück zu geben.
	\item Auf der Homepage stellt der Verein eine „Skippermappe“ zusammen, der die wesentlichen Dokumente entnommen werden können.	
\end{enumerate}

\section*{Regelungen zur Durchführung der theoretischen Ausbildung}
\subsection{Kursleiter und Referenten}
\begin{enumerate}[noitemsep]
	\item Für die jeweils verantwortliche Organisation einzelner oder mehrerer Kurse benennt der Vorstand jeweils einen Kursleiter/eine Kursleiterin als Beisitzer im erweiterten Vorstand.
		\begin{enumerate}[noitemsep]
			\item Dieser/diese ist/sind verantwortlich für die inhaltliche Festlegung und zeitliche Abstimmungder jeweiligen Kurse unter den einzelnen Referenten mit ihren jeweiligen Themenbereichen.
			\item Sie sorgen für die geeigneten Räume. Daher sind sie als Mieter oder
			beauftragter Vertreter des Mieters (Vorstand) Hausherr in den Schulungsräumen während der
			Mietzeit mit allen Rechten und Pflichten (Hausverbot u. ä.).
			\item Sie helfen bei der Beschaffung von Material. Material darf nicht mit Gewinn weitergegeben
			(verkauft) werden, da sonst die Voraussetzung für die Gemeinnützigkeit des Vereins entfällt.
			\item Sie führen die Abrechnung durch oder bereiten diese für die Kasse vor.
			\item Zum Zweck der Absprache und Abklärung können sie eine vorbereitende Konferenz aller
			Referenten eines oder mehrerer Kurse einberufen.
			\item Der Kursleiter/die Kursleiterin bietet an, Gruppen von Kursteilnehmern zu den Prüfungen
			anzumelden und führt dann die Anmeldung durch.
			\item Kursleiter können aufgrund ihrer Aufgaben zu jeder Schulung ihres Ausbildungsganges
			anwesend sein.
			\item Bei auftauchenden Unstimmigkeiten mit und zwischen Kursteilnehmern/innen oder
			Referenten ziehen sie das durch den Geschäftsverteilungsplan zuständige Vorstandsmitglied
			hinzu.
			\item Im Interesse des Vereins achten die Kursleiter darauf, dass die Schulungen im Sinne der
			Satzung und der Geschäftsordnung des Vereins erfolgen.
			\item Ein Wechsel des Referenten oder ein Thementausch kann nur mit Zustimmung des jeweiligen
			Kursleiters erfolgen.
			\item Der Einsatz von vereinsfremden Referenten bedarf im Sinne der Satzung ebenso der
			ausdrücklichen Zustimmung des jeweiligen Kursleiters. Darüber ist im jährlichen
			Geschäftsbericht gesondert zu informieren.
			\item Die Verwendung der vereinseigenen Schulungsunterlagen ist ausschließlich für die
			Schulungsmaßnahmen der HSGE bestimmt. Aus urheberrechtlichen Gründen ist eine
			Vervielfältigung und Verteilung in den Kursen nur mit ausdrücklicher Zustimmung des
			verantwortlichen Vorstandsmitglieds laut Geschäftsverteilungsplan erlaubt. (Es gibt kleinere
			Basispapiere zur Begleitung durch die Kurse.)
			\item Ein Verkauf ist wegen des Erhalts der Gemeinnützigkeit nicht möglich.
		\end{enumerate}
	\item Aufgaben der Referenten/innen:
		\begin{enumerate}[noitemsep]
			\item Referenten und Referentinnen werden vom Kursleiter mit Ausbildungseinheiten betraut.
			\item Sie bereiten in Abstimmung mit dem Kursleiter und den anderen Referenten ihre
			Kurseinheiten im Rahmen des Ausbildungsplanes vor.
			\item Bei eigenen Konzepten achten die Referenten darauf, dass sie keine
			Urheberrechtsverletzungen begehen.
			\item Sie beteiligen sich an der inhaltlichen Weiterentwicklung des Ausbildungsplanes.
			\item Sie sorgen für eine fundierte theoretische Ausbildung nach den Vorgaben der HSGE, des DSV
			oder des DMYV und ihrer Prüfungsausschüsse.
			\item Sie nehmen die weitergeleiteten Informationen über Änderungen in der Ausbildung und den
			Prüfungen zur Kenntnis und arbeiten sie in ihre Unterrichtskonzepte mit ein. Sie informieren
			Kursleiter und Referenten, sobald sie Kenntnis von Änderungen in dieser Hinsicht von anderer
			Seite erhalten.
			\item Schulungsmaterial dessen Verwertungsrecht bei der HSGE liegt, ist nach Beendigung der
			Schulungen an diese vollständig zurück zu geben. Die Weitergabe von Kopien davon an Dritte
			ist nicht gestattet.
		\end{enumerate}
\end{enumerate}

\section*{Teil 3 – Ständige Hinweise}
\setcounter{subsection}{0}
\subsection{Charterverträge}
\begin{enumerate}[noitemsep]
	\item Bei Charterverträgen ist folgendes zu beachten:
		\begin{enumerate}[noitemsep]
		\item Nie die gesetzliche Haftung des Schiffseigners übernehmen, die Skipperhaftpflicht
			zahlt dafür nicht. Formulierungen dieser Art sind in den Verträgen oder den AGB ́s zu
			streichen!
			\item In den Verträgen muss sinngemäß stehen:\\
				„Für Charterschiffschäden - einschließlich Verlust - haftet der Charterer nur, wenn
			kein Versicherungsschutz besteht und ihm Vorsatz oder grobe Fahrlässigkeit
			vorzuwerfen ist.“
		\end{enumerate}
	\item Nie den Verdienstausfall des Vercharterers übernehmen, das ist dessen unternehmerisches
	Risiko. Der Vorstand der HSGE E.V. bietet gerne seine Unterstützung bei Fragen zu
	Charterverträgen an.
	\item Bei einer Havarie haben Skipper und Crew alles zu tun, um den Schaden möglichst zu
	begrenzen. Muss geschleppt werden, ist ein sogenannter „offener Vertrag“ abzuschließen. Die
	Kosten werden dann von der Versicherung im Nachhinein ausgehandelt.
	\item Haftpflichtansprüche gegenüber Skippern auf HSGE-Törns sind unverzüglich, das heißt
	innerhalb von 3 Tagen nach dem Eintritt des Schadens oder dem Erhalt einer Nachricht über
	Forderungen gegen den Skipper oder Charterer an den Vorstand weiter zu leiten.
	Versicherungsnehmer ist die HSGE e.V.
\end{enumerate}

\section*{Teil 4}
\setcounter{subsection}{0}

\subsection{Schlussbestimmungen}
\begin{enumerate}[noitemsep]
	\item Widersprechen einzelne Regelungen der Geschäftsordnung der Satzung, so sind sie
	unwirksam, ohne dass die gesamte Geschäftsordnung ihre Wirksamkeit verliert.
	\item Diese Geschäftsordnung wurde von der Mitgliederversammlung am 17. Februar 2005 neu
	beschlossen. Zuletzt geändert durch die Mitgliederversammlung am 21. März 2019.
\end{enumerate}
\begin{flushright}
	Erlangen, den 21. März 2019\\
	\vspace{1cm}
	Wolfgang Sörgel\\
	1. Vorsitzender HSGE e.V.\\
	
\end{flushright}
\end{document}