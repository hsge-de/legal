\documentclass[10pt, twocolumn, parskip=half]{scrartcl}
\usepackage{fontspec}
\usepackage{polyglossia}
\usepackage{soul}

\usepackage{lastpage}
\usepackage{scrpage2}
\usepackage{geometry}
\usepackage{setspace}

\usepackage{tabu}
\usepackage{booktabs}

\usepackage{enumitem}

\usepackage{microtype}

\usepackage{titlesec}
%\usepackage{paralist}

\geometry{
	a4paper,
	left=1.5cm,
	right=1.5cm,
	top=3cm,
	bottom=3cm
}

\usepackage[defaultlines=2,all]{nowidow}
\usepackage{xifthen}

%\usepackage{lineno}
%\linenumbers

\setmainlanguage{german}
\setdefaultlanguage[spelling=new, babelshorthands=true]{german}
\usepackage{csquotes}

\usepackage{graphicx}

%\setmainfont[Numbers={Proportional,Lining}]{EB Garamond}
\setmainfont{Source Serif Pro}
\setsansfont{Source Sans Pro}
\setmonofont{Source Code Pro}

\onehalfspacing

\usepackage{titlesec}

\titleformat{\section}{\normalfont\large\sffamily\bfseries}{\S\space\thesection}{1em}{}

\setlength{\parindent}{0pt}

\title{\large Satzung der Hochschulsegelgruppe Erlangen e. V.}
\date{\vspace{-5ex}}

\renewcommand{\labelenumi}{(\theenumi)}
\renewcommand{\labelenumii}{(\theenumii)}

\begin{document}
	
%\maketitle
\twocolumn[{\Large\bfseries\sffamily Satzung der Hochschulsegelgruppe Erlangen e. V.} \vspace{1cm}\\\textit{Der Verein wurde am 04.08.1997 unter VR 1327 ins Vereinsregister des Amtsgerichts
Erlangen eingetragen. Diese Neufassung der Satzung wurde am 19. März 2015 von
der Mitgliederversammlung beschlossen und in das Vereinsregister eingetragen.}\vspace{0.5cm}\\]

\section*{Präambel}
 Die HSGE, Hochschulsegelgruppe Erlangen, bildete sich um
das Jahr 1979 aus segelbegeisterten Studenten und Angehörigen der Universität. Die
Mitglieder dieser lockeren Vereinigung segelten in den Semesterferien und nahmen dabei immer wieder Freunde und Studenten auf ihren Törns mit, um auch ihnen die Schönheiten des Segelns zu vermitteln.

Aus diesen Ansätzen entstand in Zusammenarbeit mit der Universität ein regelmäßiger
Schulungsbetrieb, der zur Gründung der «Hochschul-Skippervereinigung Erlangen e. V.», im
Jahr 1996 führte. Im Jahre 1999 nannte sich der Verein nach seinem Ursprung in «HSGE e. V. Hochschulsegelgruppe Erlangen e. V.» um.

Alle Bezeichnungen von Personengruppen werden in der im Deutschen üblichen
grammatikalischen männlichen Form benutzt, meinen aber sowohl weibliche als auch
männliche Personen.


\section{Name und Sitz}
\begin{enumerate}[noitemsep]
	\setlength\itemsep{0em}
	\item Der Verein trägt den Namen Hochschulsegelgruppe Erlangen e. V. (HSGE e. V.).
	\item Sitz des Vereins ist Erlangen.
	\item Der Verein soll in das Vereinsregister eingetragen sein.
	\item Der Verein wird Mitglied im Bayerischen Seglerverband.
	\item Der Verein wird Mitglied im Deutschen Seglerverband.
\end{enumerate}

\section{Geschäftsjahr}
\begin{enumerate}[noitemsep]
	\item Das Geschäftsjahr ist das Kalenderjahr.
\end{enumerate}

\section{Zweck und Ziel}
\begin{enumerate}[noitemsep]
	\item Zweck des Vereins ist die Förderung des Segelsports insbesondere durch Ausbildung, Schulung und Weiterbildung von Seglern auf der Grundlage des Amateurgedankens für Erwachsene und Jugendliche.
	\item Er fördert und pflegt auch alle Formen des Segelns als Freizeit- und Breitensport, sowie als Leistungssport und Fahrtensegeln auf See- und Binnengewässern.
	\item Der Verein fühlt sich dem Hochschulsport verbunden.
	\item Er fördert die Ausbildung, Weiterbildung und Qualifizierung von Skippern und Ausbildungsskippern im Rahmen dieser Aufgaben. Zu diesem Zweck führt er auch regelmäßig theoretische Kurse und die dazu gehörige praktische Segelausbildung durch und setzt dafür ehrenamtliche Mitarbeiter und Übungsleiter aus dem Verein ein. Näheres regelt die Geschäftsordnung.
	\item Der Verein verfolgt keine eigenwirtschaftlichen Zwecke.
	\item Die Mittel des Vereins dürfen nur für die satzungsgemäßen Zwecke verwendet werden. Es darf keine Person durch Ausgaben, die dem Zweck des Vereins fremd sind, oder durch unverhältnismäßig hohe Vergütungen begünstigt werden.
	\item Vorstände können mit einer pauschalen Aufwandsentschädigung oder sonstigen Vergütung im Rahmen der rechtlichen Regelungen bezahlt werden.	
\end{enumerate}

\section{ Mitgliedschaft und Aufnahme}
\begin{enumerate}[noitemsep]
	\item Im Verein sind:
		\begin{enumerate}[noitemsep]
			\item Ordentliche Mitglieder können natürliche Personen ohne Ansehen politischer, religiöser oder weltanschaulicher Gesichtspunkte werden.
			\item Ehrenmitglieder können natürliche Personen von innerhalb und von außerhalb des Vereins werden, die sich um die Bestrebungen des Vereins besondere Verdienste erworben haben. Ehrenvorsitzende können natürliche Personen innerhalb des Vereins werden, die sich um die Bestrebungen des Vereins als Vereinsvorstand besondere Verdienste erworben haben. Sie haben das Recht an den Vorstandssitzungen teil zu nehmen.
			\item Segler in Ausbildung sind passive Mitglieder;
			\item fördernde Mitglieder können natürliche oder juristische Personen sein.
		\end{enumerate}
	\item Stimmberechtigt sind ordentliche Mitglieder und Ehrenmitglieder, sofern sie das 18. Lebensjahr vollendet haben; im Alter von 14 bis 17 Jahren sind sie im Jugendausschuss stimmberechtigt.
	\item Näheres regelt die Geschäftsordnung.
	\item Die Mitglieder dürfen in ihrer Eigenschaft als Mitglieder kein Zuwendungen aus	Mitteln der HSGE e. V. erhalten.
	\item Die Aufnahme erfolgt auf schriftlichen Antrag durch den Vorstand nach Maßgabe einer Geschäftsordnung.
	\item Für Segler in Ausbildung endet die Mitgliedschaft automatisch am Ende des laufenden Kalenderjahres. Sie können gemäß § 4, Satz 2 a Mitglieder werden.
	\item Der Vorstand kann auf Antrag beschließen, dass die Mitgliedschaft einzelner Mitglieder aus besonderen Gründen ruht. In der Zeit, in der die Mitgliedschaft ruht, kann das Mitglied keine Aufgabe im oder für den Verein übernehmen. Über die Aufhebung des Ruhens einer Mitgliedschaft entscheidet der Vorstand. Das nähere regelt die Geschäftsordnung.
\end{enumerate}

\section{Jugendsegler}
\begin{enumerate}[noitemsep]
	\item Die Kinder und Jugendlichen des Vereins sind in der Jugendabteilung zusammengeschlossen.
	\item Die Jugendabteilung führt und verwaltet sich im Rahmen dieser Satzung selbständig. Sie entscheidet auch über die Verwendung der ihr zufließenden öffentlichen Mittel in eigener Zuständigkeit und im Rahmen der mit der Mittelgewährung gegebenen Vorschriften.
	\item Die Jugendabteilung wählt den Jugendobmann, der durch die Mitgliederversammlung bestätigt wird.
	\item Die Jugendabteilung gibt sich im Rahmen dieser Satzung eine eigene Jugendordnung.
	\item Jugendmitglieder von 14 bis 17 Jahren sind in der Jugendabteilung stimmberechtigt.
\end{enumerate}

\section{Organe des Vereins}
\begin{enumerate}[noitemsep]
	\item  Die Organe des Vereins sind die Mitgliederversammlung, der Vorstand und der erweiterte Vorstand.
\end{enumerate}

\section{Mitgliederversammlung}
\begin{enumerate}[noitemsep]
	\item Die Mitgliederversammlung besteht aus den anwesenden stimmberechtigten Mitgliedern.
	\item Die Mitgliederversammlung findet jährlich, möglichst im ersten Quartal eines Geschäftsjahres, statt.
	\item Eine außerordentliche Mitgliederversammlung findet auf Beschluss des Vorstandes oder auf Antrag von mindestens 1/3 der stimmberechtigten Mitglieder statt. Die Einladung dazu muss dann spätestens nach 2 Wochen durch den Vorstand erfolgen.
	\item Die Mitgliederversammlung ist schriftlich mit einer Frist von 4 Wochen durch den Vorstand einzuberufen. Die Tagesordnung ist dabei mitzuteilen. Mitteilungen des Vorstands können auch in elektronischer Briefform erfolgen.
	Die Einladung gilt 2 Tage nach Aufgabe bei der Post oder der Versendung als zugestellt.
	\item Wird die Einladung in einem Rundschreiben des Vereins veröffentlicht, so gilt sie 7 Tage nach dessen Verschickung als zugestellt. Die gleiche Frist gilt für alle elektronisch zugestellten Schriftstücke.
	\item Anträge von Mitgliedern, die bis 7 Tage vor der Mitgliederversammlung dem
	Vorstand zugeleitet wurden, müssen auf der Mitgliederversammlung behandelt werden.
	\item Initiativanträge während der Mitgliederversammlung sind möglich, sofern der Versammlungsleiter oder die Mitgliederversammlung der Behandlung mit einfacher Mehrheit zustimmt.
	\item Die Mitgliederversammlung ist beschlussfähig ohne Rücksicht auf die Anzahl der erschienenen Mitglieder, sofern alle stimmberechtigten Mitglieder ordnungsgemäß zur Mitgliederversammlung eingeladen worden sind.
	\item Beschlüsse werden mit der Mehrheit der anwesenden stimmberechtigten Mitglieder gefasst (mehr als 50\%).
	\item Wird bei Personenwahlen diese Mehrheit nicht erreicht, findet eine Stichwahl statt zwischen den beiden Bewerbern mit der höchsten Stimmenzahl.
	\item Die Satzung kann mit einer Mehrheit von zwei Dritteln der abgegebenen Stimmen, die Änderung des Vereinszwecks nur mit einer Mehrheit von drei Vierteln der abgegebenen Stimmen aller anwesenden stimmberechtigten Mitglieder beschlossen werden.
\end{enumerate}

\section{Aufgaben der Mitgliederversammlung}
\begin{enumerate}[noitemsep]
	\item Die Mitgliederversammlung beschließt in allen wichtigen Angelegenheiten, sofern sie nicht dem Vorstand (§ 10) vorbehalten sind. Sie hat insbesondere folgende Aufgaben:
	\begin{enumerate}[noitemsep]
		\item Wahl des Vorstandes, des Kassierers, des Schriftführers und der Kassenprüfer;
		\item Ernennung von Ehrenmitgliedern und Ehrenvorsitzenden;
		\item Bestätigung des von der Jugendabteilung gewählten Jugendobmannes;
		\item Entlastung des Vorstandes;
		\item Beitragsfestsetzung;
		\item Entscheidungen über Ausschluss von Vereinsmitgliedern bei Widerspruch gegen die Entscheidung des Vorstandes;
		\item Satzungsänderungen, Auflösung des Vereins.
	\end{enumerate}
	\item Die Beschlüsse der Mitgliederversammlung sind zu protokollieren. Das Protokoll ist vom Protokollführer und vom 1. Vorsitzenden zu unterschreiben. Es ist den Mitgliedern in geeigneter Form zur Kenntnis zu bringen.
\end{enumerate}

\section{Vorstand und erweiterter Vorstand}
\begin{enumerate}[noitemsep]
	\item Der Vorstand besteht aus:
	\begin{enumerate}[noitemsep]
		\item dem 1. Vorsitzenden
		\item bis zu zwei 2. Vorsitzenden
	\end{enumerate}
	\item Der Verein wird gerichtlich und außergerichtlich vertreten durch die Mitglieder des Vorstands und zwar von jedem einzeln. Näheres im Innenverhältnis regelt eine Geschäftsordnung.
	\item Scheidet ein Vorstandsmitglied innerhalb einer Amtszeit aus, so kann sein Amt für die Zeit bis zur nächsten ordentlichen Mitgliederversammlung kommissarisch durch ein anderes Vorstandsmitglied mitverwaltet werden.
	Der erweiterte Vorstand besteht aus den Vorsitzenden, dem Kassierer, dem
	Schriftführer, dem Jugendobmann und den Beisitzern, die vom Vorstand berufen
	werden.
	\item Vorstand, erweiterter Vorstand und die Kassenprüfer werden von der Mitgliederversammlung jeweils auf 2 Jahre gewählt. (2) Die Kassenprüfer dürfen kein Amt im Vorstand oder erweiterten Vorstand ausüben.
	\item Der Schriftführer ist für die Protokolle verantwortlich, der Kassierer verwaltet die Kasse des Vereins.
\end{enumerate}

\section{Aufgaben des Vorstandes}
\begin{enumerate}[noitemsep]
	\item Dem Vorstand obliegen insbesondere folgende Aufgaben:
	\begin{enumerate}[noitemsep]
		\item  Führung der laufenden Geschäfte des Vereins.
		\item Aufnahme und Ausschluss von Mitgliedern, Beschluss über das Ruhen einer Mitgliedschaft
		\item Bildung von Ausschüssen nach eigenem Ermessen
		\item Einberufung der Mitgliederversammlung
	\end{enumerate}
	\item Die Vorstandsmitglieder sind von den Beschränkungen des §181 BGB befreit. Diese Befreiung gilt sowohl im Innen- als auch im Außenverhältnis.
\end{enumerate}

\section{Mitgliedsbeitrag}
\begin{enumerate}[noitemsep]
	\item Der Beitrag wird durch die Mitgliederversammlung festgesetzt.
	\item Der Beitrag ist jeweils am ersten Tage des Geschäftsjahres fällig.
\end{enumerate}

\section{Ende der Mitgliedschaft}
\begin{enumerate}[noitemsep]
	\item Die Mitgliedschaft endet durch Tod, bei juristischen Personen durch Auflösung oder Änderung der Rechtsform, in der sie betrieben wird, durch Austritt oder durch Ausschluss.
	\item Der Austritt ist nur zum Ende eines Geschäftsjahres möglich und muss mindestens zwei Monate vorher schriftlich erklärt werden.
	\begin{enumerate}[noitemsep]
		\item Der Ausschluss erfolgt durch Beschluss des Vorstandes.
		\item Auf Antrag kann ausgeschlossen werden, wer in grober Weise die Satzung des Vereins verletzt oder dem Vereins- oder Verbandsinteresse entgegenarbeitet.
		\item Dies ist insbesondere der Fall bei Nichtbezahlung der Mitgliedsbeiträge nach zweimaliger schriftlicher Aufforderung.
		\item In diesem Fall endet die Mitgliedschaft mit der Zustellung der dritten schriftlichen Mitteilung zum Ende des laufenden Geschäftsjahres.
		\item Der Ausgeschlossene hat das Recht, dagegen beim Vorstand Einspruch zu erheben.
		\item Der Beschluss auf Ausschluss und der Einspruch dagegen müssen spätestens der nächsten ordentlichen Mitgliederversammlung zur endgültigen Entscheidung vorgelegt werden.
		\item Die Mitgliederversammlung entscheidet darüber mit einfacher Mehrheit der anwesenden Mitglieder.
		\item Hat ein Vereinsmitglied gegen den ihn betreffenden Beschluss des Vorstandes auf Ausschluss Einspruch eingelegt, kann es so lange nicht für oder im Namen des Vereins tätig werden, bis die Mitgliederversammlung darüber entschieden hat.
	\end{enumerate}
\end{enumerate}

\section{Auflösung des Vereins}
\begin{enumerate}[noitemsep]
	\item  Die Auflösung des Vereins kann nur mit einer Mehrheit von drei Vierteln aller Mitglieder beschlossen werden.
	\item Ist die Mitgliederversammlung zweimal hintereinander nicht beschlussfähig, kann in einer dritten Mitgliederversammlung der Beschluss auf Auflösung des Vereins mit drei Vierteln der anwesenden Mitglieder gefasst werden. Der § 13 der Satzung muss in den Einladungen dazu zitiert werden.
	\item Bei Auflösung oder Aufhebung des Vereins fällt das Vermögen des Vereins der «Deutschen Gesellschaft zur Rettung Schiffbrüchiger» zu, die es ausschließlich für gemeinnützige Zwecke zu verwenden hat. Beschlüsse über das Vermögen in diesem Zusammenhang dürfen erst nach Einwilligung der zuständigen Finanzbehörde ausgeführt werden.
\end{enumerate}

\end{document}
